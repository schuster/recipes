\documentclass{article}

\usepackage{fullpage}
\usepackage{parskip}

\title{Chili Mac}
\date{}

\begin{document}

\maketitle

Source: Cook's Illustrated, Spring 2010, p. 36

\section*{Ingredients}

\begin{itemize}
\item ½ pound elbow macaroni (about 2 cups)
\item 3 tablespoons vegetable oil
\item 1 ½ pounds 85% lean ground beef
\item 2 medium onions, chopped medium
\item 1 red bell pepper, stemmed, seeded, and chopped medium
\item 6 garlic cloves, minced
\item 2 Tbsps chili powder
\item 1 Tbsp cumin
\item 1 28oz can tomato puree
\item 1 14.5 oz can diced tomatoes
\item 1 Tbsp brown sugar (light or dark)
\item 8 oz Colby Jack cheese, shredded (about 2 cups)
\end{itemize}

\section*{Steps}

\begin{enumerate}
\item Adjust oven rack to middle position and preheat oven to 400 degrees (unless preparing ahead of time; see note below).
\item Bring 4 quarts of water to a boil, then cook macaroni and 1 Tbsp salt until al dente (about 5 minutes).
Reserve about ¾ cup pasta water, drain pasta, and set aside.
\item Add 1 Tbsp oil to pot and turn heat to medium.
When oil is shimmering, add beef and cook until brown, about 5-8 minutes.
Drain beef and set aside.
\item Add remaining 2 Tbsps oil to pot and return to medium heat until shimmering.
Add onions, red pepper, garlic, chili powder, and cumin.
Cook, stirring occasionally, until vegetables are softened and beginning to brown (about 7 minutes).
\item Add tomato puree, diced tomatoes, brown sugar, reserved pasta water, and drained beef.
Bring to simmer and cook, stirring occasionally, until flavors have blended (about 20 minutes).
\item Stir in cooked pasta and season with salt and pepper to taste.
Pour mixture into 9x13 casserole dish and smooth with spatula.
\item Sprinkle dish with cheese and cover tightly with foil.
Bake until mixture is hot and bubbling (about 20-25 minutes).
Remove foil and continue to bake until cheese begins to brown, about 5-10 minutes.
\end{enumerate}

\subsection*{To prepare ahead of time:}

After step 6, wrap dish with plastic wrap and poke holes in the wrap.
Refrigerate until cool, then wrap with another layer of plastic wrap and store for up to 2 days.
The dish can also be frozen, tightly covered with foil, up to 2 months.
Thaw in refridgerator before baking.
To cook, preheat the oven as in step 1 and follow 7, increasing cooking time to 40-45 minutes.

\end{document}
